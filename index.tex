\documentclass{beamer}
\mode<presentation>
\author{Ryan Watkins}
\title{Introduction to ReasonML}

\begin{document}

\begin{frame}
  \titlepage
\end{frame}

\begin{frame}
  \section{Javascript so far}
  \frametitle{Javascript so far}
  \pause
  \begin{columns}
    \column{.5\textwidth}
    \begin{block}{Features}
      \begin{itemize}
      \item ES6 (Babel + Webpack)
        \pause
      \item Template literals
        \pause
      \item Destructuring (single-branch pattern-matching)
        \pause
      \item Promises + (async/await)
        \pause
      \item Spread operator
        \pause
      \item Get/set on class definitions (some OOP)
      \end{itemize}
    \end{block}
    \column{.5\textwidth}
    \begin{block}{Problems}
      \begin{itemize}
      \item No great pattern matching
        \pause
      \item Non staticly typed, i.e. you get typed errors at runtime
        \pause
      \item Module system not great, you have to take care of it yourself
      \end{itemize}
    \end{block}
  \end{columns}
\end{frame}

\begin{frame}
  \section{What is ReasonML?}
  \frametitle{What is ReasonML?}
  \pause
  \begin{itemize}
  \item It's Ocaml with JS-like syntax
    \pause
  \item Contains pattern-matching
    \pause
  \item Several FP features (currying by default)
    \pause
  \item Solid type system (30+ years of research)
    \pause
  \item Extremely fast compilation to JS (10x faster than TypeScript)
  \end{itemize}
\end{frame}

\begin{frame}
  \section{Ocaml to JS?}
  \frametitle{Ocaml to JS?}
  \pause
  \begin{itemize}
  \item Made possible by bucklescript
    \pause
  \item Human readable output
    \pause
  \item Ocaml to Native and JS
    \pause
  \item But we are not writing Ocaml, ReasonML is a javascript-like syntax of Ocaml.
  \end{itemize}
\end{frame}

\begin{frame}
  \section{Reason Primitives}
  \frametitle{Reason Primitives}
  \pause
    \begin{tabular}{ | l | p{5cm} |}
      \hline
      Primitive & Example \\ \hline
      Strings & `test` \\ \hline
      Characters & 'c' \\ \hline
      Integers & 23, -23 \\ \hline
      Floats & 23.0, -23.0 \\
      \hline
    \end{tabular}
\end{frame}

\begin{frame}
  \section{Pattern Matching}
  \frametitle{Pattern Matching}
  \pause
    \begin{tabular}{ | l | p{5cm} |}
      \hline
      Javascript already has single-branch pattern matching \\ \hline
      I.e. const {foo, bar, baz} = props \\ \hline
    \end{tabular}
\end{frame}

\begin{frame}
  \section{Comparing Reason/Bucklescript to Elm}
  \frametitle{Comparing Reason/Bucklescript to Elm}
  \pause
  \begin{itemize}
  \item From the FAQ (paraphrased)
    \pause
  \item TL;DR. Elm is pure, while Reason is pragmatic (support for React).
    \pause
  \end{itemize}
\end{frame}

\begin{frame}
  \section{References}
  \frametitle{References / more useful information}
  \pause
    \begin{tabular}{ | l | p{5cm} |}
      \hline
      \item http://reasonmlhub.com/exploring-reasonml/ (web-book)
    \end{tabular}
\end{frame}


\end{document}

%%% Local Variables:
%%% mode: latex
%%% TeX-master: t
%%% End:
