\documentclass{beamer}
\mode<presentation>
\author{Ryan Watkins}
\title{Introduction to ReasonML}

\begin{document}

\begin{frame}
  \titlepage
\end{frame}


\section{Javascript so far}

\AtBeginSection[]{}
{
  \begin{frame}
    \frametitle{Outline}
    \tableofcontents[currentsection]
  \end{frame}
}

\begin{frame}
  \frametitle{Javascript so far}
  \begin{columns}
    \column{.5\textwidth}
    \begin{block}{Features}
      \begin{itemize}
      \item<2-> ES6 (Babel + Webpack)
      \item<3-> Template literals
      \item<4-> Destructuring (single-branch pattern-matching)
      \item<5-> Promises + (async/await)
      \item<6-> Spread operator
      \item<7-> Get/set on class definitions (some OOP)
      \end{itemize}
    \end{block}
    \column{.5\textwidth}
    \begin{block}{Problems}
      \begin{itemize}
      \item<8-> No great pattern matching
      \item<9-> Non staticly typed, i.e. you get typed errors at runtime
      \item<10-> Module system not great, you have to take care of it yourself
      \end{itemize}
    \end{block}
  \end{columns}
\end{frame}

\section{What is ReasonML?}
\begin{frame}
  \frametitle{What is ReasonML?}
  \pause
  \begin{itemize}
  \item<2-> It's Ocaml with JS-like syntax
  \item<3-> Contains pattern-matching
  \item<4-> Several FP features (currying by default)
  \item<5-> Solid type system (30+ years of research)
  \item<6-> Extremely fast compilation to JS (10x faster than TypeScript)
  \end{itemize}
\end{frame}


\section{Ocaml to JS?}
\begin{frame}
  \frametitle{Ocaml to JS?}
  \pause
  \begin{itemize}
  \item<2-> Made possible by bucklescript
  \item<3-> Human readable output
  \item<4-> Ocaml to Native and JS
  \item<5-> But we are not writing Ocaml, ReasonML is a javascript-like syntax of Ocaml.
  \end{itemize}
\end{frame}

\section{Reason Primitives}
\begin{frame}
  \frametitle{Reason Primitives}
    \begin{tabular}{ | l | p{5cm} |}
      \hline
      Primitive & Example \\ \hline
      Strings & `test` \\ \hline
      Characters & 'c' \\ \hline
      Integers & 23, -23 \\ \hline
      Floats & 23.0, -23.0 \\
      \hline
    \end{tabular}
\end{frame}

\section{Pattern Matching}
\begin{frame}
  \frametitle{Pattern Matching}
  \begin{itemize}
  \item<2-> Javascript already has single-branch pattern matching
  \item<3-> I.e. const {foo, bar, baz} = props
  \end{itemize}
\end{frame}

% TODO : Add some pattern matching examples


\section{Comparing Reason to Elm}
\begin{frame}
  \frametitle{Comparing Reason to Elm}
  \begin{itemize}
  \item<2-> From the FAQ (paraphrased)
  \item<3-> TL;DR. Elm is pure, while Reason is pragmatic (support for React).
  \end{itemize}
\end{frame}

\section{References}
\begin{frame}
  \frametitle{References / more useful information}
  \begin{tabular}{ | l | p{5cm} |}
    % \begin{itemize}
    % \item http://reasonmlhub.com/exploring-reasonml/ (web-book)
    % \end{itemize}
  \end{tabular}
\end{frame}


\end{document}

%%% Local Variables:
%%% mode: latex
%%% TeX-master: t
%%% End:
